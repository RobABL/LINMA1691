\documentclass[11pt,a4paper]{article}

% French
\usepackage[utf8x]{inputenc}
\usepackage[frenchb]{babel}
\usepackage[T1]{fontenc}
\usepackage{lmodern}
\usepackage{url}

% Math symbols
\usepackage{amsmath}
\usepackage{amssymb}
\usepackage{amsthm}
\usepackage{subfigure} %Allows to have several figures on the same line.
\usepackage{hyperref} %Allows to make references (\ref{}), pdf links are now clickable.
\usepackage{fmtcount} %Allows to use counters
\usepackage{fourier-orns} % Allows to display the \danger symbol
\usepackage{here} % Allows to place a figure where we want

\usepackage{makeidx} %Allows to create an index
\usepackage{enumerate} %Used where??
\makeindex
\usepackage[totoc]{idxlayout} %Allows to add the index in the table of contents

% Theorem and definitions
\theoremstyle{definition}
\newtheorem{mydef}{Définition}[subsection]
\newtheorem{mynota}[mydef]{Notation}
\newtheorem{myprop}[mydef]{Propriétés}
\newtheorem{myrem}[mydef]{Remarque}
\newtheorem{myform}[mydef]{Formules}
\newtheorem{mycorr}[mydef]{Corrolaire}
\newtheorem{mytheo}[mydef]{Théorème}
\newtheorem{mylem}[mydef]{Lemme}
\newtheorem{myexem}[mydef]{Exemple}
\newtheorem{myalgo}[mydef]{Algorithme}

\usepackage{algorithm}
\usepackage{algorithmic}

\newcommand{\bigoh}{\mathcal{O}}

\usepackage{tkz-graph}
\usepackage{tikz}
\usetikzlibrary{arrows,matrix,decorations.pathreplacing,positioning,chains,fit,shapes,calc} %Voir 1.3.8

\definecolor{mygreen}{rgb}{0,0.6,0}
\definecolor{mygray}{rgb}{0.5,0.5,0.5}
\definecolor{mymauve}{rgb}{0.58,0,0.82}

\tikzstyle{vertex}=[circle,fill=gray!50,minimum size=15pt,inner sep=0pt]
\tikzstyle{visited}=[circle,fill=green!25,minimum size=15pt,inner sep=0pt]
\tikzstyle{unvisited}=[circle,fill=blue!25,minimum size=15pt,inner sep=0pt]

\newcommand{\W}{\ {\color{red} \textbf{!!}} \ }

% Flots
\newcommand{\flotmax}{f_\mathrm{max}}
\DeclareMathOperator{\fnet}{f_\mathrm{net}}
\DeclareMathOperator{\coupe}{capacité}
\DeclareMathOperator{\coupemin}{\mathrm{coupe}_\mathrm{min}}
\DeclareMathOperator{\degin}{\deg_\mathrm{in}}
\DeclareMathOperator{\degout}{\deg_\mathrm{out}}
\DeclareMathOperator{\valeur}{valeur}
\DeclareMathOperator{\voisin}{Voisin}
