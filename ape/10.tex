\section{Séance 10}

\subsection{Flot maximum}
Déterminez le flot maximum pour le réseau ci-dessous. Comment montrer que la solution proposée est bien optimale?

\begin{figure}[h!]
  \centering
  \begin{tikzpicture}
    \SetGraphUnit{3}
    \GraphInit[vstyle=Dijkstra]
    \SetUpEdge[style={->},
    labelstyle = {sloped,draw}]
    \SetVertexNoLabel
    \Vertex[NoLabel=false]{S}
    \NOEA(S){A} \SOEA(A){O} \SOEA(S){B}
    \NOEA(O){C} \SOEA(O){D} \EA(O){E}
    \EA[NoLabel=false](E){T}
    \Edge[label=$6$](S)(A)
    \Edge[label=$4$](S)(B)
    \Edge[label=$9$](S)(O)
    \Edge[label=$3$](A)(C)
    \Edge[label=$6$](B)(D)
    \Edge[label=$1$](C)(O)
    \Edge[label=$8$](C)(T)
    \Edge[label=$1$](D)(E)
    \Edge[label=$6$](D)(T)
    \Edge[label=$1$](E)(B)
    \Edge[label=$2$](E)(C)
    \Edge[label=$4$](E)(T)
    \Edge[label=$1$](O)(D)
    \Edge[label=$8$](O)(E)
    \tikzset{EdgeStyle/.append style = {bend right}}
    \Edge[label=$2$](A)(B)
    \Edge[label=$1$](B)(A)
    \tikzset{EdgeStyle/.append style = {bend left}}
    \Edge[label=$2$](B)(C)
  \end{tikzpicture}
\end{figure}

\begin{solution}
  On voit que $f_\mathrm{net}(S) = -f_\mathrm{net}(T) = 17$.
  Si on essaie de trouver un chemin $f$-augmentant avec un BFS ou DFS
  en s'autorisant donc à prendre,
  \begin{itemize}
    \item Soit les arêtes non $f$-saturées,
    \item Soit les arêtes telles qu'il y ait une arête dans l'autre
      sens non $f$-nulles (back edges),
  \end{itemize}
  on part de $S$ mais on n’arrive jamais à $T$.
  On a donc $f_\mathrm{max} = 17$.
  \begin{center}
    \begin{tikzpicture}
      \SetGraphUnit{3}
      \GraphInit[vstyle=Dijkstra]
      \SetUpEdge[style={->},
      labelstyle = {sloped,draw}]
      \SetVertexNoLabel
      \Vertex[NoLabel=false]{S}
      \NOEA(S){A} \SOEA(A){O} \SOEA(S){B}
      \NOEA(O){C} \SOEA(O){D} \EA(O){E}
      \EA[NoLabel=false](E){T}
      \Edge[label=$5/6$](S)(A)
      \Edge[label=$4/4$](S)(B)
      \Edge[label=$8/9$](S)(O)
      \Edge[label=$3/3$](A)(C)
      \Edge[label=$5/6$](B)(D)
      \Edge[label=$0/1$](C)(O)
      \Edge[label=$7/8$](C)(T)
      \Edge[label=$0/1$](D)(E)
      \Edge[label=$6/6$](D)(T)
      \Edge[label=$1/1$](E)(B)
      \Edge[label=$2/2$](E)(C)
      \Edge[label=$4/4$](E)(T)
      \Edge[label=$1/1$](O)(D)
      \Edge[label=$7/8$](O)(E)
      \tikzset{EdgeStyle/.append style = {bend right}}
      \Edge[label=$2/2$](A)(B)
      \Edge[label=$0/1$](B)(A)
      \tikzset{EdgeStyle/.append style = {bend left}}
      \Edge[label=$2/2$](B)(C)
    \end{tikzpicture}
  \end{center}
\end{solution}


\subsection{Mon patron a tort}
Soit le réseau représenté plus bas. Votre patron est convaincu qu'il est possible de faire passer 138 unités de flots de $s$ à $t$ et il vous reproche de ne pas être capable d'exhiber un tel flot. Convainquez votre patron par un argument simple qu'un tel flot n'existe pas. Quelle est la valeur maximale d'un flot dans ce réseau?

\begin{figure}[h!]
  \centering
  \begin{tikzpicture}
    \SetGraphUnit{3}
    \GraphInit[vstyle=Dijkstra]
    \SetUpEdge[style={->},
    labelstyle = {sloped,draw}]
    \SetVertexNoLabel
    \Vertex{O}
    \WE[NoLabel=false](O){S}
    \NO(O){A} \SO(O){B} \EA(O){o}
    \NO(o){a} \SO(o){b}
    \EA[NoLabel=false](o){T}
    \Edge[label=$40$](S)(A)
    \Edge[label=$30$](S)(B)
    \Edge[label=$70$](S)(O)
    \Edge[label=$50$](A)(a)
    \Edge[label=$40$](B)(b)
    \Edge[label=$24$](B)(o)
    \Edge[label=$20$](O)(a)
    \Edge[label=$30$](O)(o)
    \Edge[label=$20$](O)(B)
    \Edge[label=$22$](a)(o)
    \Edge[label=$36$](a)(T)
    \Edge[label=$48$](o)(T)
    \Edge[label=$60$](b)(T)
    \tikzset{EdgeStyle/.append style = {bend left}}
    \Edge[label=$15$](A)(O)
    \Edge[label=$15$](O)(A)
    \Edge[label=$12$](o)(b)
    \Edge[label=$18$](b)(o)
  \end{tikzpicture}
\end{figure}

\begin{solution}
  On sait par la dualité que pour toute coupe $S \to \bar{S}$,
  $\flotmax \leq \coupe(S\to\bar{S})$.
  En prenant dans $S$ les noeuds bleus et dans $\bar{S}$ les
  noeuds verts, $\coupe(S\to\bar{S})$ vaux la somme
  des capacités des arêtes rouges, c'est-à-dire 136.
  On a donc $\flotmax \leq 136$.
  \begin{center}
    \centering
    \begin{tikzpicture}
      \SetGraphUnit{3}
      \GraphInit[vstyle=Dijkstra]
      \SetUpEdge[style={->},
      labelstyle = {sloped,draw}]
      \SetVertexNoLabel
      \Vertex{O}
      \WE[NoLabel=false](O){S}
      \NO(O){A} \SO(O){B} \EA(O){o}
      \NO(o){a} \SO(o){b}
      \EA[NoLabel=false](o){T}
      \Edge[label=$40$](S)(A)
      \Edge[label=$30$](S)(B)
      \Edge[label=$70$](S)(O)
      \Edge[label=$50$](A)(a)
      \Edge[label=$40$,color=red](B)(b)
      \Edge[label=$24$](B)(o)
      \Edge[label=$20$](O)(a)
      \Edge[label=$30$](O)(o)
      \Edge[label=$20$](O)(B)
      \Edge[label=$22$](a)(o)
      \Edge[label=$36$,color=red](a)(T)
      \Edge[label=$48$,color=red](o)(T)
      \Edge[label=$60$](b)(T)
      \tikzset{EdgeStyle/.append style = {bend left}}
      \Edge[label=$15$](A)(O)
      \Edge[label=$15$](O)(A)
      \Edge[label=$12$,color=red](o)(b)
      \Edge[label=$18$](b)(o)
      \AddVertexColor{blue}{S,A,B,O,a,o}
      \AddVertexColor{green}{b,T}
    \end{tikzpicture}
  \end{center}

  Soit $f$ le flot du graph suivant,
  $\valeur(f) = 136$.
  On sait donc que
  $136 = \valeur(f) \leq \flotmax \leq 136$.
  On a donc nécessairement $\flotmax = 136$.
  \begin{center}
    \centering
    \begin{tikzpicture}
      \SetGraphUnit{3}
      \GraphInit[vstyle=Dijkstra]
      \SetUpEdge[style={->},
      labelstyle = {sloped,draw}]
      \SetVertexNoLabel
      \Vertex{O}
      \WE[NoLabel=false](O){S}
      \NO(O){A} \SO(O){B} \EA(O){o}
      \NO(o){a} \SO(o){b}
      \EA[NoLabel=false](o){T}
      \Edge[label=$40/40$](S)(A)
      \Edge[label=$30/30$](S)(B)
      \Edge[label=$62/70$](S)(O)
      \Edge[label=$40/50$](A)(a)
      \Edge[label=$40/40$](B)(b)
      \Edge[label=$10/24$](B)(o)
      \Edge[label=$18/20$](O)(a)
      \Edge[label=$22/30$](O)(o)
      \Edge[label=$20/20$](O)(B)
      \Edge[label=$22/22$](a)(o)
      \Edge[label=$36/36$](a)(T)
      \Edge[label=$48/48$](o)(T)
      \Edge[label=$46/60$](b)(T)
      \tikzset{EdgeStyle/.append style = {bend left}}
      \Edge[label=$0/15$](A)(O)
      \Edge[label=$0/15$](O)(A)
      \Edge[label=$0/12$](o)(b)
      \Edge[label=$6/18$](b)(o)
    \end{tikzpicture}
  \end{center}
\end{solution}

\subsection{Entremetteuses de couples ardennais}
Dans un petit village en Ardenne, il y a $n$ célibataires masculins, $n$ célibataires féminins et $m$ entremetteuses. Chaque entremetteuse connait certains des célibataires masculins ainsi que certains des célibataires féminins. De plus, l'entremetteuse $i$ ne peut arranger qu'au plus $b_i$ mariages entre les célibataires qu'elle connait. On suppose que seuls les mariages hétérosexuels sont acceptés et que les célibataires ne se marient qu'une fois. On souhaite déterminer le nombre maximum de mariages qui peuvent être arrangés. Montrez comment ce problème peut être formulé comme un problème de flot maximum dans un graphe.

\begin{solution}
  On peut modéliser ce problème comme un flot maximum dans un graphe de $2n + 2m + 2$ noeuds donc
  \begin{itemize}
    \item une source,
    \item un puits,
    \item les $n$ célibataires masculins,
    \item les $n$ célibataires féminins et
    \item les $m$ entremetteuses dupliquées,
      c'est la façon standard modéliser une capacité sur un noeud et non sur une arête.
      On duplique le noeud en question, une première partie avec toutes les arêtes entrantes
      et une deuxième partie avec toutes les arêtes sortantes.
      Une arête avec la capacité du noeud comme capacité est alors ajoutée entre les deux parties.
  \end{itemize}
  Pour modéliser le fait qu'ils ne peuvent se marier qu'une seule fois, on relie tous les célibataires
  masculins (resp. féminins) à une source (resp. un puits) avec une capacité 1.
  La capacité entre les célibataires et les entremetteuses n'a alors pas trop d'importance,
  elle doit juste être plus grande que 1.

  Le flot maximum du graphe est alors le nombre de mariages maximum.
  Un exemple est illustré ci-dessous avec $n = 4$, $m = 3$ et des connaissances quelconques.
  \begin{center}
    \begin{tikzpicture}[x=2cm,y=1cm]
      \SetGraphUnit{1}
      \GraphInit[vstyle=Dijkstra]
      \SetUpEdge[style={->},
      labelstyle = {draw}]
      \Vertex{S}
      \NOEA(S){M2}
      \SOEA(S){M3}
      \NOEA(M2){E1}
      \SOEA(M2){E2}
      \SOEA(M3){E3}
      \NOWE(E1){M1}
      \SOWE(E3){M4}
      \EA(E1){E1'}
      \EA(E2){E2'}
      \EA(E3){E3'}
      \NOEA(E1'){F1}
      \SOEA(E1'){F2}
      \SOEA(E2'){F3}
      \SOEA(E3'){F4}
      \SOEA(F2){T}
      \Edge[label=$1$](S)(M1)
      \Edge[label=$1$](S)(M2)
      \Edge[label=$1$](S)(M3)
      \Edge[label=$1$](S)(M4)
      \Edge[label=$1$](M1)(E1)
      \Edge[label=$1$](M2)(E1)
      \Edge[label=$1$](M2)(E2)
      \Edge[label=$1$](M3)(E3)
      \Edge[label=$1$](M4)(E3)
      \Edge[label=$b_1$](E1)(E1')
      \Edge[label=$b_2$](E2)(E2')
      \Edge[label=$b_3$](E3)(E3')
      \Edge[label=$1$](E1')(F1)
      \Edge[label=$1$](E2')(F1)
      \Edge[label=$1$](E3')(F3)
      \Edge[label=$1$](E3')(F2)
      \Edge[label=$1$](F1)(T)
      \Edge[label=$1$](F2)(T)
      \Edge[label=$1$](F3)(T)
      \Edge[label=$1$](F4)(T)
    \end{tikzpicture}
  \end{center}
\end{solution}

\subsection{Processeurs bicoeurs}
Vous disposez d'une machine parallèle de deux processeurs A et B de types différents sur laquelle vous devez faire tourner une série de $n$ processus. Chaque processus doit être attribué à un processeur. Si vous choisissez d'assigner le processus $i$ au processeur A, cela induit un coût $\alpha_i$, alors que si vous choisissez le B, le coût est $\beta_i$. De plus, il est avantageux de faire tourner certains processus sur le même processeur parce que les transferts de données entre les deux sont élevés. Le coût d'assigner les processus $i$ et $j$ à des processeurs différents est égal à $c_{ij}$. Déterminez l'attribution optimale pour les données ci-dessous.

\begin{multicols}{2}

\begin{center}
\begin{tabular}{||c||c|c|c|c||}
\hline
$i$ & 1 & 2 & 3 & 4 \\
\hline
$\alpha_i$ & 6 & 5 & 10 & 4 \\
\hline
$\beta_i$ & 4 & 10 & 3 & 8 \\
\hline
\end{tabular}
\end{center}

\columnbreak

\begin{center}
\begin{tabular}{||c||c|c|c|c||}
\hline
$c_{ij}$ & 1 & 2 & 3 & 4 \\
\hline
1 & 0 & 5 & 0 & 0 \\
\hline
2 & 5 & 0 & 6 & 2 \\
\hline
3 & 0 & 6 & 0 & 1 \\
\hline
4 & 0 & 2 & 1 & 0 \\
\hline
\end{tabular}
\end{center}

\end{multicols}

\begin{solution}
  Ça peut se représenter comme un Min Cut qui peut également
  être résolu par Ford-Fulkerson, car $\coupemin = \flotmax$.

  En effet, prenons le graphe avec 6 noeuds.
  Une source $A$, un puits $B$ et 4 noeuds représentant les 4 processus.
  On ajoute
  \begin{itemize}
    \item Une arête de capacité $c_{ij}$ du processus $i$ au processus $j$,
      si $c_{ij} \neq 0$ (pour simplifier, car les arêtes de poids 0 n'influencent pas $\flotmax$ et $\coupemin$).
    \item Une arête de capacité $\alpha_i$ de $A$ au processus $i$.
    \item Une arête de capacité $\beta_i$ du processus $i$ à $B$.
  \end{itemize}
  Si on veut couper le graphe en 2 avec d'un côté, $A$ et les processus $b$ qui sont assignés à $B$
  et d'un autre $B$ et les processus $a$ qui sont assignés à $A$
  On doit couper $\beta_a$, $\alpha_b$ et $c_{ba}$.
  $\coupemin$ sur ce graphe nous donnera donc la valeur minimum de $\sum_a \beta_a + \sum_b \alpha_b + \sum_b \sum_a c_{ba}$.

  Dans notre cas, le graphe est le suivant.
  \begin{center}
    \begin{tikzpicture}[x=2cm,y=1cm]
      \SetGraphUnit{1.5}
      \GraphInit[vstyle=Dijkstra]
      \SetVertexMath
      \Vertex{A}
      \EA(A){2}
      \NO(2){1}
      \SO(2){3}
      \SO(3){4}
      \EA(3){B}
      \SetUpEdge[style={->}]
      labelstyle = {draw}]
      \Edge[label=$6$](A)(1)
      \Edge[label=$5$](A)(2)
      \Edge[label=$10$](A)(3)
      \Edge[label=$4$](A)(4)
      \Edge[label=$4$](1)(B)
      \Edge[label=$10$](2)(B)
      \Edge[label=$3$](3)(B)
      \Edge[label=$8$](4)(B)
      \SetUpEdge[style={<->}]
      \tikzset{EdgeStyle/.append style = {bend right}}
      \Edge[label=$5$](1)(2)
      \Edge[label=$6$](2)(3)
      \Edge[label=$1$](3)(4)
      \tikzset{EdgeStyle/.append style = {bend left}}
      \Edge[label=$2$](2)(4)
    \end{tikzpicture}
  \end{center}

  En appliquant l'algorithme de Ford-Fulkerson, on trouve le graphe suivant
  avec la coupe en rouge.
  On observe donc qu'il faut assigner les processus 1, 2 et 3 à $B$
  et le processus 4 à $A$ pour un coût de 22.
  \begin{center}
    \begin{tikzpicture}[x=2cm,y=1cm]
      \SetGraphUnit{2}
      \GraphInit[vstyle=Classic]
      \SetVertexMath
      \SetUpVertex[Lpos=90]
      \Vertex{A}
      \EA(A){2}
      \NOEA(2){1}
      \SOEA(2){3}
      \SOWE(3){4}
      \EA(3){B}
      \SetUpEdge[style={->},
      labelstyle = {draw}]
      \Edge[label=$5/6$](A)(1)
      \Edge[label=$5/5$](A)(2)
      \Edge[label=$10/10$](A)(3)
      \Edge[label=$4/4$,color=red](A)(4)
      \Edge[label=$4/4$,color=red](1)(B)
      \Edge[label=$10/10$,color=red](2)(B)
      \Edge[label=$3/3$,color=red](3)(B)
      \Edge[label=$7/8$](4)(B)
      \Edge[label=$0/5$](2)(1)
      \Edge[label=$0/6$](2)(3)
      \tikzset{EdgeStyle/.append style = {bend right}}
      \Edge[label=$1/1$,color=red](3)(4)
      \Edge[label=$1/5$](1)(2)
      \Edge[label=$2/2$](2)(4)
      \Edge[label=$6/6$](3)(2)
      \Edge[label=$0/1$](4)(3)
      \Edge[label=$0/2$](4)(2)
      \AddVertexColor{blue}{A,1,2,3}
      \AddVertexColor{green}{4,B}
    \end{tikzpicture}
  \end{center}
\end{solution}

\subsection{Algorithme de Ford-Fulkerson}
L'algorithme de Ford-Fulkerson consiste à trouver à chaque itération un chemin d'augmentation et d'augmenter le flot le long de ce chemin jusqu'à saturation. Montrez par un argument élémentaire que dans le cas où les capacités maximales sont entières cet algorithme s'arrête toujours après un nombre fini d'itérations. Pouvez-vous adapter votre argument à la situation pour laquelle les capacités sont rationnelles? \\
\textbf{Challenge:} et pour des capacités réelles?

\begin{solution}
  \begin{itemize}
    \item
      Considérons $\fnet(S)$, le flot net sortant des sources.
      À chaque itération, il augmente de la capacité du chemin d'augmentation.
      Comme les capacités sont entières et que la capacité d'un chemin d'augmentation est strictement positive,
      elle est au moins égale à 1.
      Après $n$ itérations, on a donc $\fnet(S) \geq n$.
      On sait que ce flot net est au maximum la somme des capacités des arêtes sortant des sources, notons-la $\mathcal{C}$,
      c'est à dire $\fnet(S) \leq \mathcal{C}$.
      On a donc $n \leq \mathcal{C}$. Comme $\mathcal{C}$ est fini, l'algorithme va s'arrêter après un nombre fini d'itérations.
    \item
      Si elles sont rationnelles, on peut les écrire sous forme d'une fraction d'entiers.
      Soit $p$, le PPCM des dénominateurs de ces fractions.
      Si on multiplie toutes les capacités par $p$, ça ne change pas le nombre d'itérations
      et on arrive à un cas avec que des capacités entières.
      L'algorithme va donc aussi s'arrêter après un nombre fini d'opérations.
    \item
      Pour des capacités réelles, il est possible que l'algorithme ne s'arrête jamais comme c'est le cas du
      graphe suivant avec $r = \frac{\sqrt{5} - 1}{2}$.
      \begin{center}
        \begin{tikzpicture}
          \SetGraphUnit{2}
          \GraphInit[vstyle=Dijkstra]
          \SetUpEdge[style={->},
          labelstyle = {draw}]
          \Vertex[L=$s$]{S}
          \SOWE[L=$v_1$](S){v1} \SO[L=$v_2$](S){v2} \SOEA[L=$v_3$](S){v3}
          \EA[L=$v_4$](v3){v4}
          \SO[L=$t$](v3){T}
          \Edge[label=$2$](S)(v1)
          \Edge[label=$2$](S)(v2)
          \Edge[label=$2$](S)(v4)
          \Edge[label=$1$](v2)(v1)
          \Edge[label=$1$](v2)(v3)
          \Edge[label=$r$](v4)(v3)
          \Edge[label=$2$](v1)(T)
          \Edge[label=$2$](v3)(T)
          \Edge[label=$2$](v4)(T)
        \end{tikzpicture}
      \end{center}

      Il se peut que Ford-Fulkerson
      trouve les chemins augmentant dans l'ordre suivant,
      rappelons-nous que $r^2 = 1 - r$ et donc aussi $r^{n+2} = r^n - r^{n+1}$.
      \begin{center}
        \begin{tabular}{|c|c|c|c|c|}
          \hline
          Chemin augmentant & Flot du chemin & $v_2v_1$ & $v_2v_3$ & $v_4v_3$\\
          \hline
          & & 1 & 1 & $r$\\
          \hline
          $s, v_2, v_3, t$ & 1 & 1 & 0 & $r$\\
          \hline
          $s, v_4, v_3, v_2, v_1, t$ & $r$ & $r^2$ & $r$ & 0\\
          \hline
          $s, v_2, v_3, v_4, t$ & $r$ & $r^2$ & 0 & $r$\\
          \hline
          $s, v_4, v_3, v_2, v_1, t$ & $r^2$ & 0 & $r^2$ & $r^3$\\
          \hline
          $s, v_1, v_2, v_3, t$ & $r^2$ & $r^2$ & 0 & $r^3$\\
          \hline
        \end{tabular}
      \end{center}
      On vérifie par récurrence que le flot après $2n+1$ itérations suivant
      ce schéma de choix de chemins augmentant vaut $1 + 2\sum_{i=1}^n r^i$
      et que ce schéma ne dépasse jamais la capacité des arêtes.

      Le flot maximum donné par le graphe résiduel suivant est 5
      \begin{center}
        \begin{tikzpicture}
          \SetGraphUnit{2}
          \GraphInit[vstyle=Dijkstra]
          \SetUpEdge[style={->},
          labelstyle = {draw}]
          \Vertex[L=$s$]{S}
          \SOWE[L=$v_1$](S){v1} \SO[L=$v_2$](S){v2} \SOEA[L=$v_3$](S){v3}
          \EA[L=$v_4$](v3){v4}
          \SO[L=$t$](v3){T}
          \Edge[label=$2/2$](S)(v1)
          \Edge[label=$1/2$](S)(v2)
          \Edge[label=$2/2$](S)(v4)
          \Edge[label=$0/1$](v2)(v1)
          \Edge[label=$1/1$](v2)(v3)
          \Edge[label=$0/r$](v4)(v3)
          \Edge[label=$2/2$](v1)(T)
          \Edge[label=$1/2$](v3)(T)
          \Edge[label=$2/2$](v4)(T)
        \end{tikzpicture}
      \end{center}
      Seulement,
      \begin{align*}
        1 + 2\sum_{i=1}^\infty r^i & = 1 + 2r\frac{1-r^\infty}{1-r}\\
                                   & = 1 + 2\frac{r}{r^2}\\
                                   & = 1 + 2\frac{1}{r}\\
                                   & = 1 + 4\frac{1}{\sqrt{5}-1}\\
                                   & = 1 + 4\frac{\sqrt{5}+1}{4}\\
                                   & = \sqrt{5}+2\\
                                   & < 5.
      \end{align*}
      On atteindra donc jamais le flot maximum.
  \end{itemize}
\end{solution}

\subsection{Profit d'un laboratoire}
Un laboratoire a la possibilité de réaliser 6 projets. La réalisation du projet $i$ génère un revenu $q_i$. Pour réaliser le projet $i$, le laboratoire a besoin d'une présence de l'ensemble des chercheurs $T_i$. Engager le chercheur $j$ coûte $p_j$. Étant donné les revenus $q= (8,7,4,5,3,9)$, les coûts $p=(2,10,7,3,5)$ et $T_1 = \left\lbrace 1, 4 \right\rbrace$, $T_2 = \left\lbrace 1,2,3,4 \right\rbrace$, $T_3 = \left\lbrace 1,3,5 \right\rbrace$, $T_4 = \left\lbrace 4,5 \right\rbrace$, $T_5 = \left\lbrace 3,4,5 \right\rbrace$, $T_6 = \left\lbrace 1, 4,5 \right\rbrace$, trouvez l'ensemble des projets qui maximise le profit total (revenus moins coûts).

\begin{solution}
  On peut résoudre ce problème avec un MaxFlow comportant des coûts négatifs.
  Il faudra alors trouver les chemins $f$-augmentants à l'aide un algorithme de plus court chemin insensible
  aux cycles de poids négatifs tel Belmann-Ford ou SPFA (Shortest Path Fast Algorithm).

  Il existe cependant une solution plus élégante à ce problème ne comportant pas d'arête
  de poids négatif.
\end{solution}

\paragraph{Exercices supplémentaires}

\subsection{Acheminer des produits}
Un producteur désire envoyer au départ des sommets $D_1$ et $D_2$ un produit aux destinations $M_1, M_2$, et $M_3$ à travers le réseau. Les capacités des arêtes sont limitées. Il y a des demandes respectives de 10, 8 et 8 unités aux destinations $M_1, M_2,$ et $M_3$. Ces demandes peuvent-elles être satisfaites?

\begin{figure}[h!]
  \centering
  \begin{tikzpicture}[x=2cm,y=1cm]
    \SetGraphUnit{1.5}
    \GraphInit[vstyle=Dijkstra]
    \SetUpEdge[style={->},
    labelstyle = {draw}]
    \SetVertexNoLabel
    \Vertex[NoLabel=false]{D1}
    \NOEA(D1){A} \SOEA(D1){B}
    \SOWE[NoLabel=false](B){D2}
    \SOEA(D2){C}
    \NOEA(B){D} \SOEA(B){E}
    \NOEA[NoLabel=false](D){M1}
    \SOEA[NoLabel=false](D){M2}
    \SOEA[NoLabel=false](E){M3}
    \Edge[label=$5$](D1)(D2)
    \Edge[label=$20$](D1)(A)
    \Edge[label=$5$](D2)(B)
    \Edge[label=$6$](D2)(C)
    \Edge[label=$5$](A)(B)
    \Edge[label=$10$](A)(D)
    \Edge[label=$6$](B)(D)
    \Edge[label=$5$](B)(E)
    \Edge[label=$5$](C)(B)
    \Edge[label=$5$](C)(E)
    \Edge[label=$4$](D)(E)
    \Edge[label=$20$](D)(M1)
    \Edge[label=$15$](E)(M3)
    \Edge[label=$6$](M1)(M2)
    \Edge[label=$6$](M3)(M2)
  \end{tikzpicture}
\end{figure}

\begin{solution}
  Il nous suffit d'ajouter un puits relié à M1, M2 et M3 avec les demandes respectives.
  Si $\flotmax = 10+8+8=26$, les demandes peuvent être satisfaites,
  sinon elles ne le peuvent pas.
  Ça donne le réseau suivant
  \begin{center}
    \begin{tikzpicture}[x=2cm,y=1cm]
      \SetGraphUnit{1.5}
      \GraphInit[vstyle=Dijkstra]
      \SetUpEdge[style={->},
      labelstyle = {draw}]
      \SetVertexNoLabel
      \Vertex[NoLabel=false]{D1}
      \NOEA(D1){A} \SOEA(D1){B}
      \SOWE[NoLabel=false](B){D2}
      \SOEA(D2){C}
      \NOEA(B){D} \SOEA(B){E}
      \NOEA[NoLabel=false](D){M1}
      \SOEA[NoLabel=false](D){M2}
      \SOEA[NoLabel=false](E){M3}
      \EA[NoLabel=false](M2){T}
      \Edge[label=$5$](D1)(D2)
      \Edge[label=$20$](D1)(A)
      \Edge[label=$5$](D2)(B)
      \Edge[label=$6$](D2)(C)
      \Edge[label=$5$](A)(B)
      \Edge[label=$10$](A)(D)
      \Edge[label=$6$](B)(D)
      \Edge[label=$5$](B)(E)
      \Edge[label=$5$](C)(B)
      \Edge[label=$5$](C)(E)
      \Edge[label=$4$](D)(E)
      \Edge[label=$20$](D)(M1)
      \Edge[label=$15$](E)(M3)
      \Edge[label=$6$](M1)(M2)
      \Edge[label=$6$](M3)(M2)
      \Edge[label=$10$](M1)(T)
      \Edge[label=$8$](M2)(T)
      \Edge[label=$8$](M3)(T)
    \end{tikzpicture}
  \end{center}

  On voit clairement que le flot maximum est 26 en prenant $\bar{S} = \{T\}$.
  Il existe d'ailleurs un flot de valeur 26 donné par le graphe suivant
  \begin{center}
    \begin{tikzpicture}[x=2cm,y=1cm]
      \SetGraphUnit{1.5}
      \GraphInit[vstyle=Dijkstra]
      \SetUpEdge[style={->},
      labelstyle = {draw}]
      \SetVertexNoLabel
      \Vertex[NoLabel=false]{D1}
      \NOEA(D1){A} \SOEA(D1){B}
      \SOWE[NoLabel=false](B){D2}
      \SOEA(D2){C}
      \NOEA(B){D} \SOEA(B){E}
      \NOEA[NoLabel=false](D){M1}
      \SOEA[NoLabel=false](D){M2}
      \SOEA[NoLabel=false](E){M3}
      \EA[NoLabel=false](M2){T}
      \Edge[label=$0/5$](D1)(D2)
      \Edge[label=$15/20$](D1)(A)
      \Edge[label=$5/5$](D2)(B)
      \Edge[label=$6/6$](D2)(C)
      \Edge[label=$5/5$](A)(B)
      \Edge[label=$10/10$](A)(D)
      \Edge[label=$6/6$](B)(D)
      \Edge[label=$5/5$](B)(E)
      \Edge[label=$1/5$](C)(B)
      \Edge[label=$5/5$](C)(E)
      \Edge[label=$0/4$](D)(E)
      \Edge[label=$16/20$](D)(M1)
      \Edge[label=$10/15$](E)(M3)
      \Edge[label=$6/6$](M1)(M2)
      \Edge[label=$2/6$](M3)(M2)
      \Edge[label=$10/10$](M1)(T)
      \Edge[label=$8/8$](M2)(T)
      \Edge[label=$8/8$](M3)(T)
    \end{tikzpicture}
  \end{center}

\end{solution}

\subsection{Représentant de clubs de sport}
On considère un groupe de six personnes $\left\lbrace A, B, \ldots, F \right\rbrace$ qui sont toutes membres d'un certain nombre de clubs de sport $\left\lbrace 1, 2, 3, 4, 5, 6 \right\rbrace$. Ci-dessous nous donnons la liste des membres de chaque club.
\begin{description}
  \item [Club 1] $A, C$
  \item [Club 2] $C, E$
  \item [Club 3] $A, B, C, D, E, F$
  \item [Club 4] $A, C, E$
  \item [Club 5] $A, B, D, E, F$
  \item [Club 6] $A, E$
\end{description}
On souhaite choisir un représentant pour chaque club.
Pour être représentant d'un club, il faut en être membre.
Par ailleurs, une même personne ne peut représenter qu'un seul club.
Est-il possible de trouver des représentants pour tous les clubs ?
Justifiez votre réponse.
Proposez une solution dans le cas d'un ensemble $M$ de personnes
et un ensemble $C$ de clubs, avec $S_c \subseteq M$ la liste des membres du club $c \in C$.

\begin{solution}
  C'est un problème de couplage maximum dans un graphe biparti (Maximum Cardinality Bipartite Matching (MCBM)).
  Si le nombre d'arêtes du couplage est égal au nombre de club,
  on pourra trouver un représentant pour chaque club.

  On pourrait utiliser le théorème de Hall et prendre tous les sous-ensembles $E \subseteq C$
  et vérifier que $\voisin(E) \geq |E|$.
  Mais ça fait $2^6 = 64$ ensembles et donc il vaut mieux utiliser Ford-Fulkerson.

  Dans le cas particulier du MCBM,
  l'algorithme de Fork-Fulkerson se simplifie grandement, mais contentons-nous d'appliquer l'algorithme
  général pour le Max Flow.

  On a $|M|$ noeuds représentant les personnes,
  $|C|$ noeuds représentant les clubs, une source $s$ et un puits $t$.
  Il y a
  \begin{itemize}
    \item une arête de $m \in M$ à $c \in C$ si $m \in S_c$ car il faut être membre pour être représentant d'un club,
    \item une arête de $s$ à $m \in M$ de capacité 1, car chaque personne ne peut être représentant d'un seul club,
    \item une arête de $c \in C$ à $t$ de capacité 1, car chaque club n'a au maximum qu'un seul représentant.
  \end{itemize}
  La capacité des arêtes de $m \in M$ à $c \in C$ n'importe pas tant qu'il est plus grand que 1,
  car on a déjà limité la capacité avec la source et le poids.
  Donnons une capacité unitaire également pour faire simple.

  Dans notre cas, ça forme le graphe suivant,
  comme les arêtes sont toutes de capacité unitaire,
  on a omis le poids pour plus de clarté.
  \begin{center}
    \begin{tikzpicture}[x=2cm,y=1cm]
      \SetGraphUnit{1.5}
      \GraphInit[vstyle=Classic]
      \SetVertexMath
      \SetUpVertex[Lpos=90]
      \Vertex{1}
      \SO(1){2}
      \SO(2){3}
      \SO(3){4}
      \SO(4){5}
      \SO(5){6}
      \begin{scope}
        \SetGraphUnit{3} \EA(1){A}
      \end{scope}
      \SO(A){B}
      \SO(B){C}
      \SO(C){D}
      \SO(D){E}
      \SO(E){F}
      \WE(3){s}
      \EA(D){t}
      \SetUpEdge[style={->}]
      \Edge(s)(1)
      \Edge(s)(2)
      \Edge(s)(3)
      \Edge(s)(4)
      \Edge(s)(5)
      \Edge(s)(6)
      \Edge(1)(A)
      \Edge(1)(C)
      \Edge(2)(C)
      \Edge(2)(E)
      \Edge(3)(A)
      \Edge(3)(B)
      \Edge(3)(C)
      \Edge(3)(D)
      \Edge(3)(E)
      \Edge(3)(F)
      \Edge(4)(A)
      \Edge(4)(C)
      \Edge(4)(E)
      \Edge(5)(A)
      \Edge(5)(B)
      \Edge(5)(D)
      \Edge(5)(E)
      \Edge(5)(F)
      \Edge(6)(A)
      \Edge(6)(E)
      \Edge(A)(t)
      \Edge(B)(t)
      \Edge(C)(t)
      \Edge(D)(t)
      \Edge(E)(t)
      \Edge(F)(t)
    \end{tikzpicture}
  \end{center}

  En appliquant Ford-Fulkerson, on remarque qu'on
  ne peut pas trouver un représentant pour chaque club.
  Le couplage maximum a seulement 5 arêtes.

  Le flot maximum $f$ est représenté par le graphe suivant dans lequel les arêtes rouges sont
  les arêtes utilisées et donc $f$-saturées car leur capacité est 1.
  On remarque que si on cherche un chemin $f$-augmenté,
  c'est-à-dire qu'on part de $s$ et qu'on ne peut prendre que les arêtes noires
  dans le bon sens ou les arêtes brunes dans le sens inverse,
  on sait explorer les noeuds $s,1,2,4,6,A,C,E$ mais pas les noeuds $3,5,B,D,F,t$.
  \begin{center}
    \begin{tikzpicture}[x=2cm,y=1cm]
      \SetGraphUnit{1.5}
      \GraphInit[vstyle=Classic]
      \SetVertexMath
      \SetUpVertex[Lpos=90]
      \Vertex{1}
      \SO(1){2}
      \SO(2){3}
      \SO(3){4}
      \SO(4){5}
      \SO(5){6}
      \begin{scope}
        \SetGraphUnit{3} \EA(1){A}
      \end{scope}
      \SO(A){B}
      \SO(B){C}
      \SO(C){D}
      \SO(D){E}
      \SO(E){F}
      \WE(3){s}
      \EA(D){t}
      \SetUpEdge[style={->}]
      \Edge(s)(4)
      \Edge(1)(C)
      \Edge(2)(E)
      \Edge(3)(A)
      \Edge(3)(C)
      \Edge(3)(D)
      \Edge(3)(E)
      \Edge(3)(F)
      \Edge(4)(A)
      \Edge(4)(C)
      \Edge(4)(E)
      \Edge(5)(A)
      \Edge(5)(B)
      \Edge(5)(E)
      \Edge(5)(F)
      \Edge(6)(A)
      \Edge(F)(t)
      \SetUpEdge[style={->},color=brown]
      \Edge(s)(1)
      \Edge(s)(2)
      \Edge(s)(3)
      \Edge(s)(5)
      \Edge(s)(6)
      \Edge(1)(A)
      \Edge(2)(C)
      \Edge(3)(B)
      \Edge(5)(D)
      \Edge(6)(E)
      \Edge(A)(t)
      \Edge(B)(t)
      \Edge(C)(t)
      \Edge(D)(t)
      \Edge(E)(t)
    \end{tikzpicture}
  \end{center}
\end{solution}
