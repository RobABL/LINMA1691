\section{Séance 6}
\textbf{Coloriages d'arêtes}

\subsection{Indice chromatique du graphe de Pétersen}
Déterminez l'indice chromatique du graphe de Pétersen.

\subsection{Tournoi d'échecs}
Dans un tournoi d'échecs, chaque engagé doit rencontrer tous les autres. Chaque partie dure une heure. Déterminez la durée minimum du tournoi dans le cas où le nombre d'engagés est 3, 4, ou 5.

\subsection{Carré latin}
Un carré latin est une matrice $n \times n$ dans laquelle les entrées d'une colonne (ou d'une ligne) particulière sont toutes distinctes. Formulez ce problème comme un problème de coloriage d'arêtes et donnez le nombre minimum de symboles qui permettent de satisfaire cette contrainte.
