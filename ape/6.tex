\section{Séance 6}

\paragraph{Couplages}

\subsection{Couplage maximum}
Etant donné le couplage $M = \left\lbrace  (1,1'), (2,4'),(4,2'),(5,5')  \right\rbrace$ dans le graphe biparti ci-dessous, démontrez que $M$ est maximum.

\begin{figure}[h!]
  \begin{center}
    \begin{tikzpicture}[-,>=stealth',shorten >=1pt,auto]
      \Vertex[x=0 ,y=0]{1}
      \Vertex[x=0 ,y=-1]{2}
      \Vertex[x=0,y=-2]{3}
      \Vertex[x=0 ,y=-3]{4}
      \Vertex[x=0 ,y=-4]{5}
      \Vertex[x=3 ,y=-0]{1'}
      \Vertex[x=3 ,y=-1]{2'}
      \Vertex[x=3 ,y=-2]{3'}
      \Vertex[x=3 ,y=-3]{4'}
      \Vertex[x=3 ,y=-4]{5'}


      \path[every node/.style={font=\sffamily\small}]
      (1) edge [style=dashed]  node [left] {} (1')
      edge node [left] {} (5')

      (2) edge node [right] {} (1')
      edge node [right] {} (2')
      edge node [right] {} (3')
      edge [style=dashed]  node [right] {} (4')
      edge node [right] {} (5')

      (3) edge node [right] {} (1')
      edge node [left] {} (5')

      (4) edge [style=dashed]  node [right] {} (2')
      edge node [right] {} (3')
      edge node [right] {} (4')
      edge node [right] {} (5')

      (5) edge node [right] {} (1')
      edge [style=dashed]  node [left] {} (5');


    \end{tikzpicture}
  \end{center}
\end{figure}
\begin{solution}
Un couplage dans un graphe est un ensemble M d’arêtes tel que
M ne contient pas de boucles et deux arêtes de M n’ont jamais
d’extrêmité en commun. Hors, les seuls noeuds qui ne sont pas encore des extrémités dans ce couplage M sont les noeuds $3$ et $3'$. Si M est maximum, on ne peut pas lui ajouter d'arête. Hors la seule arête qu'on pourrait lui ajouter serait une qui relie les noeuds $3$ et $3'$. Cette arête n'existant pas dans ce graphe, M est bien maximum.
\end{solution}

\subsection{Vrai ou faux}

\begin{itemize}
  \item Un arbre possède un couplage parfait si et seulement si tous les chemin d'une feuille à une autre sont de longueur impaire.
  \item Un arbre possède au plus un couplage parfait.
  \item Dans un arbre, si pour tout noeud $u$, il existe une feuille $v$ telle que $d(u,v)$ est impaire, alors cet arbre possède un couplage parfait.
  \item Soit $o(H)$, le nombre de composantes impaires du graphe $H$, c'est-à-dire le nombre de composantes connexes ayant un nombre impair de sommets. Un arbre $G$ admet un couplage parfait si $o(G-v)=1,  \ \ \forall v \in V$.
  \item Si un arbre $G$ admet un couplage parfait, alors $o(G-v)=1, \ \ \forall v \in V$.
\end{itemize}


\subsection{Assistanat INMA}
Thibault a la responsabilité de répartir les séances d'exercices des cours de mathématiques appliquées entre les assistants du pôle INMA. Chacun des assistants donne pour cela à Thibault une liste de ses cours préférés, qui sont repris dans la table ci-dessous.

\begin{center}
  \begin{tabular}{|c|c|}
    \hline
    Assistant & Cours préférés \\
    \hline
    Pierre & Projet, Théorie des Matrices \\
    Romain & Graphes, Modélisation Stochastique \\
    Arnaud & Théorie des Matrices \\
    Adeline & Graphes, Optimisation, Analyse Numérique \\
    Benoit & Théorie des Matrices, Projet \\
    Nicolas & Graphes, Optimisation, Analyse Numérique, \\
            & Modélisation Stochastique, Théorie des Matrices  \\
    \hline
  \end{tabular}
\end{center}

Thibault aimerait bien assigner exactement un cours à chaque assistant en respectant autant que possible leurs préférences. Formulez cela comme un problème de couplage maximum dans un graphe.

Pour l'aider dans sa tâche, Thibault dispose de la répartition de l'année dernière:

\begin{center}
  \begin{tabular}{|c|c|}
    \hline
    Romain & Modélisation Stochastique \\
    Adeline & Optimisation \\
    Nicolas & Théorie des Matrices \\
    Pierre & Projet \\
    \hline
  \end{tabular}
\end{center}

En partant de la répartition de l'année dernière, utilisez l'algorithme hongrois pour aider Thibault à trouver un couplage maximum. Ce couplage est-il parfait? Proposez un argument pour prouver que le couplage trouvé est effectivement maximum.

\begin{solution}
Dans le graphe graphe biparti ci-dessous, un couplage possible est indicé en vert (utilisez l'algorithme hongrois pour trouver un couplage maximum). Nous avons donc un couplage maximum de 5. Il n'est pas un couplage parfait car Benoit, Pierre et Arnaud ne peuvent avoir que Théorie des Matrices ou  Projet.

\begin{center}
\begin{tikzpicture}[-,>=stealth',shorten >=1pt,auto]
   \Vertex[x=0 ,y=0]{P}
   \Vertex[x=0 ,y=-1]{R}
   \Vertex[x=0,y=-2]{Ar}
   \Vertex[x=0 ,y=-3]{Ad}
   \Vertex[x=0 ,y=-4]{B}
   \Vertex[x=0, y=-5]{N}
   \Vertex[x=3 ,y=-0]{Pr}
   \Vertex[x=3 ,y=-1]{TM}
   \Vertex[x=3 ,y=-2]{G}
   \Vertex[x=3 ,y=-3]{MS}
   \Vertex[x=3 ,y=-4]{O}
   \Vertex[x=3, y=-5]{AN}

     \path[every node/.style={font=\sffamily\small}]
     (P) edge  node [left] {} (TM)
        	edge [style=green]  node [left] {} (Pr)
        	edge [style=dashed]  node [left] {} (Pr)

    (R) edge node [right] {} (G)
    		edge[style=green] node [right] {} (MS)
    	edge [style=dashed]  node [right] {} (MS)

    (N) edge node [right] {} (G)
    		edge[style=green] node [right] {} (O)
     	edge  node [right] {} (O)
    		edge node [right] {} (AN)
    		edge node [right] {} (MS)
         edge [style=dashed]  node [left] {} (TM)

    (Ar) edge[style=green] node [right] {} (TM)
    		edge node [right] {} (TM)

     (Ad) edge[style=green] node [right] {} (G)
     	edge [style=dashed] node [right] {} (O)
         edge node [left] {} (G)
         edge node [left] {} (AN)

     (B) edge node [right] {} (TM)
         edge  node [left] {} (Pr);


\end{tikzpicture}
\end{center}
\end{solution}

\subsection{Gagner sans le couplage parfait}
Deux personnes jouent à un jeu sur un graphe $G$ de la manière suivante:

\begin{itemize}
  \item Chacune des deux personnes sélectionne chacune à son tour un sommet $v_1, v_2, v_3, …$ tel que $\forall i > 1, \ \ \ v_i$ est adjacent à $v_{i-1}$.
  \item Un sommet déjà sélectionné ne peut plus être choisi.
  \item La dernière personne à sélectionner un sommet gagne le jeu.
\end{itemize}

Montrer que le premier joueur admet une stratégie gagnante si et seulement si le graphe $G$ n'admet pas de couplage parfait.

\paragraph{Coloriages d'arêtes}

\subsection{Indice chromatique du graphe de Pétersen}
Déterminez l'indice chromatique du graphe de Pétersen.


\begin{solution}
L'indice chromatique du graphe de Petersen est de 4 comme nous pouvons le voir sur le graphe suivant:

\begin{tikzpicture}[-,>=stealth',shorten >=1pt,auto]
      \Vertex[x=0 ,y=3]{1}
      \Vertex[x=0 ,y=1]{2}
      \Vertex[x=3,y=0]{3}
      \Vertex[x=1 ,y=0]{4}
      \Vertex[x=-3 ,y=0]{5}
      \Vertex[x=-1 ,y=0]{6}
      \Vertex[x=-2 ,y=-3]{7}
      \Vertex[x=-1 ,y=-1]{8}
      \Vertex[x=2 ,y=-3]{9}
      \Vertex[x=1 ,y=-1]{10}


      \path[every node/.style={font=\sffamily\small}]

      (1) edge [color=blue] node [left] {} (3)
      edge [color=black]  node [left] {} (5)
      edge [color=red]  node [left] {} (2)
      
      (2) edge [color=blue] node [left] {} (8)
      edge [color=black]  node [left] {} (10) 
      
      (3) edge [color=red] node [left] {} (4)
      edge [color=black]  node [left] {} (9)    

	 (4) edge [color=blue] node [left] {} (6)
      edge [color=green]  node [left] {} (8) 
      
      (5) edge [color=red] node [left] {} (6)
      edge [color=blue]  node [left] {} (7)
      
      (6) edge [color=green]  node [left] {} (10)
      
      (7) edge [color=black]  node [left] {} (8)
       edge [color=red]  node [left] {} (9)
      
      (9) edge [color=blue]  node [left] {} (10);    

    \end{tikzpicture}
\end{solution}

\subsection{Tournoi d'échecs}
Dans un tournoi d'échecs, chaque engagé doit rencontrer tous les autres. Chaque partie dure une heure. Déterminez la durée minimum du tournoi dans le cas où le nombre d'engagés est 3, 4, ou 5.

\begin{solution}
Dans le problème les noeuds sont les joueurs. Nous avons donc des graphes complet de 3, 4 et 5 car chaque joueurs doit rencontré tous les autres. Le problème se résume à trouver l'indice chromatique ce ceux-ci. En effet, la couleur représentera les "round" du tournoi et aussi le nombre d'heure.

Pour $K_{3}$, $K_{4}$ et $K_{5}$, nous avons respectivement 3, 3 et 5 heures.
\end{solution}

\subsection{Carré latin}
Un carré latin est une matrice $n \times n$ dans laquelle les entrées d'une colonne (ou d'une ligne) particulière sont toutes distinctes. Formulez ce problème comme un problème de coloriage d'arêtes et donnez le nombre minimum de symboles qui permettent de satisfaire cette contrainte.
