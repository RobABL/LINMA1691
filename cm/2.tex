\section{Les plus courts chemins}
\subsection{Les plus courts chemins}
\index{graphe!graphe pondéré}
\begin{mydef}
  Une \emph{fonction de poids} sur un graphe ($V$, $E$, $\varphi$) est une fonction $E \to \mathbb{R}$. Un \emph{graphe pondéré} est un graphe muni d’une fonction de poids. Le \emph{poids} ou la \emph{longueur} d’un parcours est la somme des poids des arêtes qui le compose.
\end{mydef}

\begin{mytheo} [Plus court chemin et plus court parcours]
  Pour un graphe avec une fonction de poids $\geq 0$, si le plus court parcours entre $u$ et $v$ est de longueur $d$, alors le plus court chemin entre $u$ et $v$ est aussi de longueur $d$.
  \begin{proof}
    Soit $d'$ la longueur du plus court chemin.
    Comme un chemin est un parcours, on a nécessairement $d' \geq d$ car $d$ est la longueur du plus court parcours.
    Supposons par l'absurde qu'elle soit différente et donc que $d' > d$.
    Prenons le plus cours parcours et tant qu'il contient des cycles $v_0 \ldots v_i \ldots v_i \ldots v_n$,
    retirons $v_i \ldots v_i$ pour avoir $v_0 \ldots v_i \ldots v_n$.
    Une fois qu'il n'y aura plus de cycle, ce sera un chemin de longueur $d'' \leq d$.
    Comme $d'$ est la longueur du plus cours chemin, $d'' \geq d'$ ce qui est absurde car $d'' \leq d < d'$.
  \end{proof}
\end{mytheo}

\subsection{Algorithme de Dijkstra}
\index{algorithme!algorithme de Dijkstra}
\begin{myalgo}[Algorithme de Dijkstra]
  L'algorithme de Dijkstra donne la distance la plus courte entre un noeud $u_0$ et tous les autres
  noeuds du graphe dans un graphe connexe dont toutes les arêtes sont de poids positif.
  \begin{enumerate}
    \item Initialement, $i=0$, l'ensemble $S={u_{0}}$ (il contient notre noeud de départ) et
      l'ensemble $\bar{S}_{0} = V \setminus S$.
      $\ell(u_{0})=0$ et $\ell(v)=\infty$ pour $v\ne u_{0}$ ($\ell(v)$ est une borne supérieure de $d(u_{0},v)$).
    \item Pour chaque $v \in \bar{S}_{i}$, on remplace $\ell(v)$ par $\min(\ell(v),\ell(u_{i})+w(u_{i}v))$.
      On calcule le minimum pour $v \in \bar{S}_{i}$ de $\ell(v)$ et on nomme $u_{i+1}$ le sommet pour qui ce minimum est atteint.
      Ensuite, on fixe $S_{i+1}=S_{i} \cup \{u_{i+1}\}$ et on retire ce sommet de $\bar{S}_{i+1}$.
    \item Si $i = |V| - 1$, on arrête. Si $i < |V| - 1$, on remplace $i$ par $i+1$ et on refait l'étape 2.
  \end{enumerate}
\end{myalgo}

\begin{myexem}
  On cherche les distances à partir de $a$.\\
	\begin{center}
	  \begin{tikzpicture}
      \node[vertex] at (0,  2) (a) {$a$};
      \node[vertex] at (2,  1) (b) {$b$};
      \node[vertex] at (1, -2) (c) {$c$};
      \node[vertex] at (-1,-2) (d) {$d$};
      \node[vertex] at (-2, 1) (e) {$e$};
      \draw[->] (a) edge node[anchor = south] {\tiny $50$} (b);
  		\draw[->] (c) edge node[anchor = north] {\tiny $5$} (b);
  		\draw[->] (c) edge node[anchor = north] {\tiny $50$} (d);
  		\draw[->] (e) edge node[anchor = south] {\tiny $10$} (d);
  		\draw[->] (a) edge node[anchor = south] {\tiny $10$} (e);
  		\draw[->] (d) edge node[anchor = north] {\tiny $5$} (a);
  		\draw[->] (a) edge node[anchor = south] {\tiny $30$} (c);
  		\draw[->] (d) edge node[anchor = north] {\tiny $20$} (b);
  	\end{tikzpicture}
  \end{center}
  \begin{center}
    \begin{tabular}{|c|c|ccccc|}
      \hline
      $u'$ & $S$ & $l(a)$ & $l(b)$ & $l(c)$ & $l(d)$ & $l(e)$\\
      \hline
      $a$ & $\lbrace a \rbrace$ & $\boxed{0}$ & $\infty$ & $\infty$ & $\infty$ &$\infty$\\
      $a$ & $\lbrace a,e \rbrace$ && 50 & 30 & $\infty$ & $\boxed{10}$\\
      $e$ & $\lbrace a,e,d \rbrace$ && 50 & 30 & $\boxed{20}$ &\\
      $d$ & $\lbrace a,e,d,c \rbrace$ && 40 & $\boxed{30}$ & &\\
      $c$ & $\lbrace a,e,d,c,b \rbrace$ && $\boxed{35}$ & & &\\
      \hline
    \end{tabular}
  \end{center}
\end{myexem}

\begin{mytheo} [L'algorithme de Dijkstra fonctionne]
  Après chaque MISE À JOUR DE $\ell$ dans l'algorithme, les deux propriétés suivantes sont vérifiées :
  \begin{itemize}
    \item pour $v \in S$, $\ell(v) = d(u_0, v)$ et le chemin le plus court de $u_0$ à $v$ reste dans $S$;
    \item pour $v \notin S$, $\ell(v) \geq d(u_0, v)$, et $\ell(v)$ est la longueur du plus court chemin de $u_0$ vers $v$ dont tous les noeuds internes sont dans $S$.
  \end{itemize}
  \begin{proof}
    Montrons le par induction
    \begin{description}
      \item[Initialisation]
        Au début, $\ell(u_0) = 0 = d(u_0, u_0)$ et $\ell(v) = \infty > d(u_0, v)$
        car on considère le graphe connexe.
      \item[Induction]
        Montrons les deux items, l'un après l'autre,
        \begin{itemize}
          \item
            Montrons par l'absurde qu'après avoir ajouté un noeud $u_i$ dans $S$, on sait que $\ell(u_i) = d(u_0, u_i)$.
            Supposons que $\ell(u_i) > d(u_0, u_i)$.
            Comme $\ell(u_i)$ est la longueur du plus court chemin où tous les
            noeuds intermédiaires sont dans $S_{i-1}$,
            la seule possibilité pour que $\ell(u_i) \neq d(u_0, u_i)$,
            c'est qu'il existe un plus court chemin passant par un noeuds de $\bar{S}_{i-1}$.
            Soit $v$, le premier noeud de $\bar{S}_{i-1}$ de ce chemin.
            Le sous-chemin de $u_0$ à $v$ est le plus court chemin de $u_0$ à $v$ par
            le lemme~\ref{lem:sub}.
            Par la définition de $v$, ce chemin ne passe que par des noeuds de
            $S_{i-1}$.
            Du coup, on sait que $d(u_0, v) = \ell(v)$.
            Comme les arêtes sont de poids positifs, on a $\ell(v) = d(u_0, v) \leq d(u_0, u_i)$.
            Mais comme on a pris $u_i$ tel que $\ell(u_i) = \min_{u \in \bar{S}_{i-1}} \ell(u)$,
            on a $\ell(u_i) \leq \ell(v)$.
            Dès lors $\ell(u_i) \leq d(u_0, u_i)$, ce qui est absurde car on a supposé le contraire.
          \item
            Commençons par montrer que $\ell(v) \geq d(u_0, v)$.
            Les seuls, $\ell(v)$ qui changent changent en $\ell(u_i) + w(u_i, v)$.
            On remarque que c'est la longueur du chemin composé du plus court
            chemin de $u_0$ à $u_i$ et de l'arête de $u_i$ à $v$.
            Par la définition de $d(u_0, v)$, on a donc nécessairement l'inégalité voulue.

            Pour le dernier point,
            par l'hypothèse de récurrence, on sait que $\ell(v)$ est la longueur du plus court
            chemin de $u_0$ vers $v$ pour tout noeud de $S_{i-1}$.
            Pour que ce soit vrai pour tout noeud de $S_i$, il faut considérer les chemin passant
            par $u_i$.
            Il y a un de ces chemins ayant comme avant dernier noeud $u_i$ et pas de $u_i$ avant
            car l'avant avant dernier noeud était dans $S$ avant $u_i$ donc il a une chemin
            le plus court sans passer par $u_i$.
            Donc le seul chemin qu'il faut encore considérer est le chemin le plus court
            allant de $u_0$ à $u_i$ suivit ensuite par $v$.
            Comme on fait $\ell(v) = \min(\ell(v), \ell(u_i) + w(u_i, v))$, on en prend compte.
        \end{itemize}
    \end{description}
  \end{proof}
\end{mytheo}

\begin{mycorr} [L'algorithme de Dijkstra est correct]
  L’algorithme de Dijkstra est correct.
\end{mycorr}

\begin{mytheo} [L’algorithme de Dijkstra est quadratique]
  L’algorithme de Dijkstra sur un graphe se termine en un temps de l’ordre $n^2$ .
  \begin{proof}
    \noindent
    \begin{itemize}
      \item A chaque passage de boucle $|\bar{S}|$ décroit de 1. Ce qui implique $n$ passages de boucles.
      \item A chaque boucle, ``$\forall v \in \bar{S}$'' fait $\bigoh(n-|S|)$ opérations;
        et ``trouver $v_{min}$`` fait $\bigoh(n-|S|)$ opérations également.
      \item Cela implique un temps total $\bigoh(n + (n-1) + (n-2) + ... + 1) = \bigoh\left( \frac{n(n-1)}{2} \right) = \bigoh(n^2)$
        puisqu'on considère le temps à une constante près.
    \end{itemize}
  \end{proof}
\end{mytheo}

\index{graphe!graphe dirigé}
\begin{mydef}
  Un \emph{graphe dirigé} est un triplet ($V$, $E$, $\varphi$), où :
  \begin{itemize}
    \item $V$ est un ensemble dont les éléments sont appelés sommets ou noeuds;
    \item $E$ est un ensemble dont les éléments sont appelés arêtes;
    \item $\varphi$ est une fonction, dîte fonction d'incidence, qui associe à chaque arête un \emph{couple}
      \footnote{Dans la définition~\ref{def:graph} on a parlé de paire et ici de couple car une paire est
        un ensemble de 2 éléments alors qu'un couple est un $n$-uplet avec $n = 2$.
        Dans un ensemble, il n'y a pas de relation d'ordre alors que dans un $n$-uplet,
        chaque élément a un indice bien définit.
        Une arête est donc à sens unique dans un graphe dirigé.}
      de sommets.
  \end{itemize}
\end{mydef}

\begin{myexem}
Pour le graphe dirigé suivant :
  \begin{center}
	  \begin{tikzpicture}
      \node[vertex] at (0,  2) (a) {$a$};
      \node[vertex] at (2,  1) (b) {$b$};
      \node[vertex] at (1, -2) (c) {$c$};
      \node[vertex] at (-1,-2) (d) {$d$};
      \node[vertex] at (-2, 1) (e) {$e$};
      \draw[->] (a) edge node[anchor = south]   {\tiny $1$} (b);
  		\draw[->] (c) edge node[anchor = north] {\tiny $2$} (b);
  		\draw[->] (c) edge node[anchor = north] {\tiny $3$} (d);
  		\draw[->] (e) edge node[anchor = south] {\tiny $4$} (d);
  		\draw[->] (a) edge node[anchor = south] {\tiny $5$} (e);
  		\draw[->] (d) edge node[anchor = north] {\tiny $6$} (a);
  		\draw[->] (a) edge node[anchor = south] {\tiny $7$} (c);
  		\draw[->] (d) edge node[anchor = north] {\tiny $8$} (b);
  	\end{tikzpicture}
  \end{center}
	Nous avons :
    \begin{align*}
      V & = \left\{a,b,c,d,e\right\}\\
      E & = \left\{1,2,3,4,5,6,7,8\right\}\\
      \varphi(1) & = (a,b)\\
      \varphi(2) & = (c,b)\\
      \varphi(3) & = (c,d)\\
      \varphi(4) & = (e,d)\\
      \varphi(5) & = (a,e)\\
      \varphi(6) & = (d,a)\\
      \varphi(7) & = (a,c)\\
      \varphi(8) & = (d,b)
    \end{align*}
\end{myexem}

\subsection{Anneaux et Semi-anneaux}
\index{anneau}
\index{anneau!semi-anneau}
\begin{mydef}
  Un \emph{semi-anneau} est un ensemble muni de deux opérations ($\oplus,\otimes$) et ses propriétés.
\end{mydef}

\begin{mydef}
  Un \emph{anneau} est un ensemble muni de trois opérations ($\oplus,\otimes,\ominus$) et ses propriétés.
\end{mydef}

\begin{mydef}
  Un \emph{corps} est un ensemble muni de quatre opérations ($\oplus,\otimes,\ominus,\oslash$) et ses propriétés.
\end{mydef}

Par exemple, une propriété peux être la distributivité :
$$a\otimes (b \oplus c) = (a \otimes b)\oplus(a \otimes c) $$
On peut par exemple définir des matrices avec un semi-anneau :
\begin{eqnarray}
  A \oplus B &=& (a_{ij} \oplus b_{ij})_{ij}\\
  A \otimes B &=& (\sum_k a_{ik} \otimes b_{kj})_{ij}
\end{eqnarray}
Le $\oplus$ définit la somme matricielle et le $\otimes$ le produit matriciel.

Prenons $\oplus = \min$ et $\otimes = +$.
Le neutre de $\min$ est $\infty$ et celui de $+$ est $0$ donc
\[ I =
  \begin{pmatrix}
    0 & \infty & \ldots & \infty\\
    \infty & \ddots & \ddots & \vdots\\
    \vdots & \ddots & \ddots & \infty\\
    \infty & \ldots & \infty & 0
  \end{pmatrix}.
\]

En prenant $A$ tel que $a_{ij} = w(i, j)$, la matrice des plus court parcourt est
\[ I \oplus A \oplus \ldots \oplus A^{\otimes n} \]

Pour calculer le plus court chemin, on peut définir alors l'itération:
\begin{eqnarray}
  M_0 &=& I\\
  M_1 &=& I \oplus (M_0 \otimes A)\\
  M_2 &=& I \oplus (M_1 \otimes A)\\
   & \vdots &\\
  M_{k+1} &=& I \oplus (M_k \otimes A)
\end{eqnarray}
On a fini de converger quand $M_k = M_{k+1}$. Dijkstra peut être vu comme une manière efficace d'implémenter cette itération.
